\capitulo{6}{Trabajos relacionados}

La popularidad de los algoritmos de PLN ha provocado una enorme avalancha de proyectos y trabajos con la intención de mejorarlos, ampliarlos, estudios y comparativas de eficiencia entre diferentes configuraciones de parámetros y el uso de diferentes analizadores léxicos, y la búsqueda de nuevas áreas de aplicación. Entre los proyectos previos que podemos encontrar en la universidad en relación con esta temática se destacan los proyectos dispuestos a continuación.

\section{Label prediction on issue tracking systems using text mining} \label{sec:labelpredictionarticle}
El artículo <<\emph{Label prediction on issue tracking systems using text mining}>> \cite{ related_works:label_prediction} es el punto de partida del proyecto \textbf{GitHub Text Mining}. Este ensayo tuvo como objetivo el estudio de la clasificación de incidencias por medio del uso de avanzadas técnicas de minería de datos y clasificación como son las \textbf{Máquinas de Vectores de Soporte (SVM)} y el clasificador \textbf{Naive Bayes Multinomial (NVM)}. Sus investigaciones se centraron en el entrenamiento de los clasificadores anteriormente mencionados y el estudio de la precisión de las predicciones obtenidas sobre las incidencias un conjunto discreto de repositorios de GitHub.

A raíz de los resultados logrados surgió la idea de diseñar una plataforma que permitiese automatizar el proceso de obtención de los datos y experimentar con la aplicación modelos de clasificación preentrenados. Estos modelos preentrenados permitirían eludir el complejo proceso de entrenar un modelo desde cero, así como conseguir comprender una variedad temática más amplia a costa de cierta pérdida de precisión en los resultados. A lo largo de la evolución del proyecto se decidió ampliar el tipo de tareas de PLN soportadas por la plataforma.

\section{JIZT}
\emph{JIZT} \cite{related_works:jizt} es una plataforma de generación de resúmenes abstractivos que hace uso de técnicas de procesamiento del lenguaje natural para lograr su cometido. El proyecto logró ofrecer unos buenos resultados mediante la aplicación de modelos de PLN y la incorporación de novedosas técnicas en el preprocesado de los datos. Estas innovaciones han sido incorporadas en nuestro proyecto con el objetivo de lograr mejorar los resultados en aquellas situaciones en las que los datos sobre los cuales se deseaba aplicar los modelos superaban las limitaciones de tamaño de estos.

\section{TFG Plataforma de text mining sobre repositorios de código abierto GitLab}
\emph{Este trabajo} \cite{related_works:tfg_text_mining_gitlab} se incluye con el objetivo de señalar otro enfoque distinto mediante el cual afrontar un proyecto que tiene como base el mismo punto de partida en el \hyperref[sec:labelpredictionarticle]{artículo anteriormente mencionado}. La plataforma presentada tendría como objetivo extraer de manera automatizada de datos de repositorios de \emph{GitLab} y permitir la clasificación de las incidencias de acuerdo múltiples estrategias implementadas en específico para la plataforma.