\capitulo{1}{Introducción}

Los proyectos software requieren del establecimiento de una serie de medidas que permitan organizar y gestionar su progreso a lo largo de las diversas fases del desarrollo. Entre estas medidas en los proyectos software destacan los \textbf{sistemas de gestión y seguimiento de incidencias}. La tendencia en el uso de metodologías rápidas en este tipo de proyectos implica la necesidad de dividir los grandes proyectos en cortas iteraciones compuestas por pequeñas tareas. La asignación de este tipo de tareas suele llevarse a cabo de manera individual, debido a que de manera diaria se realizan reuniones donde exponer los problemas encontrados y proponer soluciones e ideas en conjunto.

La organización de este tipo de reuniones no siempre tiene porqué llevarse a cabo de manera presencial, es posible realizarse a través de foros internos donde debatir las propuestas, clasificar las tareas, gestionar las incidencias detectadas, y en definitiva analizar la situación de las pequeñas tareas con el objetivo de concluir el proyecto de acuerdo con los requisitos en tiempo y recursos establecidos. En este punto es donde surgen programas o plataformas con la finalidad de simplificar la gestión de las tareas en grandes proyectos, cuya organización puede resultar compleja y caótica.

El inconveniente de este tipo de plataformas recae en que pese a facilitar las labores de seguimiento del estado de los procesos o incidencias, siguen requiriendo un alto grado de atención manual por parte del encargado de su gestión. La razón detrás de ello recae en el hecho de que la definición, descripción, debate, y conclusión de las tareas poseen un alto componente social que no podía ser interpretado por las máquinas.

El punto de inflexión con los problemas anteriores llega en los últimos 10 años, momento en el cual las empresas comenzaron a descubrir el potencial de la inmensa cantidad de información que estas habían almacenado durante décadas. La necesidad de llevar a cabo un tratamiento de estos datos propició el creciente desarrollo y evolución de las \textbf{técnicas de procesamiento del lenguaje natural (PLN)}. Estas técnicas tienen como principal objetivo la aplicación de \textbf{Inteligencia Artificial (IA)} para el análisis automático de las comunicaciones humanas por medio de los computadores, permitiendo la extracción, clasificación y descubrimiento de detalles e información relevante que no aparece dispuesta de manera explícita.

Estas técnicas de procesamiento del lenguaje natural en conjunto con la necesidad de automatizar aquellas labores de seguimiento de incidencias en la gestión de un proyecto es el lugar donde se sitúa nuestro objetivo con este trabajo. Con la finalidad de comprobar y demostrar el potencial de esta especialidad del aprendizaje automático en su aplicación a las labores de seguimiento de incidencias de un proyecto software este trabajo plantea el diseño de una plataforma con la que experimentar con distintos modelos de PLN aplicables y su ajuste a través de la modificación de parámetros.  

Los modelos de PLN que se ha decidido incluir en la plataforma permiten al usuario conocer la probabilidad con la que el contenido de una incidencia se ajusta a una serie de etiquetas definidas tanto en el propio repositorio como introducidas por el usuario (\textbf{\emph{Zero-Shot Classification}}), generar resúmenes abstractivos que sinteticen el contenido de las incidencias (\textbf{\emph{Summarization}}), y obtener gráficos que señale la postura de los participantes con respecto a una incidencia deducida a partir del análisis de su actitud en los comentarios (\textbf{\emph{Sentiment Analysis}}).
