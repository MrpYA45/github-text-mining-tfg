\capitulo{7}{Conclusiones y Líneas de trabajo futuras}

\section{Conclusiones}

La principal conclusión que se puede apreciar es en el hecho de que se ha conseguido cumplir con los objetivos planteados para el proyecto. Se ha logrado obtener una aplicación funcional, la cual permite extraer información desde repositorios de GitHub y someter estos datos a modelos de procesamiento del lenguaje natural preentrenados. 

Uno de los principales motivos para acudir al uso de modelos preentrenados recaían en su principal ventaja, la posibilidad de incluir de este tipo de tecnologías de aprendizaje supervisado sin la necesidad de invertir una gran cantidad de tiempo en su planificación y modelado. El uso de modelos preentrenados tiene como objetivo servir a proyectos donde el diseño y ajuste de los modelos no es la principal motivación, sino que estos actúan como un componente más del proceso que finalmente permita obtener unos resultados. Es evidente, a raíz de las pruebas a las que se ha sometido la aplicación, que las soluciones propuestas por los experimentos podrían mejorar si se recurriese al uso de modelos entrenados en específico para la realización de tareas clasificación, análisis o generación de resúmenes sobre procesos de la ingeniería del software como es el caso de la gestión de tareas.

Con respecto a la arquitectura, se ha logrado alcanzar el objetivo de aplicar un diseño que utilice una arquitectura de microservicios de manera que se explorasen las ventajas de segmentar el código en servicios que actúan de manera colaborativa manteniendo un bajo acoplamiento. La ventaja de utilizar este tipo de arquitecturas recae en el hecho de que otorgan de una amplia disponibilidad a los sistemas, los cuales no requieren de una parada completa de los servidores para la incorporación de cambios, y dotas a las aplicaciones de una estructura más resistente ante posibles problemas o caídas de alguno de sus componentes. A su vez, la limitación de las responsabilidades de cada sección del proceso a su propio componente, facilita su mantenimiento y la escalabilidad de los recursos.

En cuanto a la variedad de técnicas de desarrollo implicadas en el proyecto, estas han permitido estudiar y comprobar las particularidades, ventajas e inconvenientes de cada una de las ramas en proyecto software. Durante la realización del proyecto se ha tratado con tecnologías de bases de datos, diseño de API REST, configuración de sistemas distribuidos, programación de servidores web, realización de pruebas, uso de herramientas de depuración y diseño de interfaces. El recurrir a tan amplio número de tecnologías ha permitido ampliar el porfolio de conocimientos adquiridos durante la carrera, así como fomentar la motivación para continuar profundizando en aquellas áreas que han despertado un mayor interés a nivel personal.

Finalmente, el desarrollo de un proyecto a largo plazo ha supuesto conocer en primera instancia el proceso de planificación y toma de decisiones requerido para sacar adelante este tipo de proyectos. Es fundamental establecer los objetivos en las etapas iniciales de manera que se pueda diseñar un plan de actuación y una distribución de las tareas ajustada a los plazos. Se deben diseñar estrategias que permitan resolver los problemas planteados por proyecto estudiando las posibilidades que ofrecen las diferentes tecnologías disponibles. 


\section{Líneas de trabajo futuras}

En esta sección se incluyen ideas acerca de la incorporación de nuevas funcionalidades que quedaron descartadas debido al alcance establecido para el proyecto, así como posibles áreas de mejora sobre las que seguir trabajando.

\subsection{Mejoras e incorporación de nuevas tareas}

Desde el inicio el proyecto GitHub Text Mining se planteó con la idea de establecer una arquitectura base que permitiese la ampliación de los modelos soportados a otro tipo de tareas. Actualmente los experimentos que se incluyen permiten obtener una clasificación de las incidencias entorno a etiquetas, generar resúmenes a partir su contenido y realizar un análisis de sentimientos sobre las incidencias, incluyendo filtros de usuarios y comentarios. En adición a la introducción de mejoras en las respectivas implementaciones de estas tareas, se plantea la introducción de nuevas funcionalidades que permitan detectar incidencias duplicadas en un repositorio haciendo uso de modelos de \textbf{Sentence Similarity}, así como estudiar la introducción de tareas de \textbf{Traducción}.

\subsection{Implementar acceso mediante el uso de tokenes de acceso personal por usuario}

Una decisión de diseño tomada inicialmente condiciona a la utilización de un único token de acceso personal para el establecimiento de la conexión con la API de GitHub. Este factor puede resultar limitante en caso de que se pretenda desplegar la aplicación a causa de las \textbf{limitaciones de la API de GitHub} \cite{lf:github_api_limits}. Sería preferible diseñar alguna estrategia alternativa en la que cada usuario deba proporcionar su propio token para poder utilizar la plataforma. Se deberá tener en cuenta que estos tokenes son información personal, y por lo tanto deberán transferirse, tratarse y almacenarse de forma segura.

\subsection{Refractorización de la API REST}

La \emph{API REST} ha ido creciendo conforme se requerían nuevas formas de acceder a los datos desde la aplicación web. Esto ha propiciado la inclusión de una gran cantidad de métodos que deberían ser refractorizados con el objetivo de reducir su complejidad y deshacerse de aquellos \emph{endpoints} que resulten prescindibles. Otro punto de mejora a considerar implicaría la inclusión de un sistema que estableciese limitaciones en el número de peticiones que se permiten recibir desde una misma dirección durante un cierto periodo de tiempo.
