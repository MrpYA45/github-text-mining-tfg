\documentclass[a4paper,12pt,twoside]{memoir}

% Castellano
\usepackage[spanish,es-tabla]{babel}
\selectlanguage{spanish}
\usepackage[utf8]{inputenc}
\usepackage[T1]{fontenc}
\usepackage{lmodern} % Scalable font
\usepackage{microtype}
\usepackage{placeins}

\RequirePackage{booktabs}
\RequirePackage[table]{xcolor}
\RequirePackage{xtab}
\RequirePackage{multirow}

% Links
\PassOptionsToPackage{hyphens}{url}\usepackage[colorlinks]{hyperref}
\hypersetup{
	allcolors = {blue}
}

% Ecuaciones
\usepackage{amsmath}

% Rutas de fichero / paquete
\newcommand{\ruta}[1]{{\sffamily #1}}

% Párrafos
\nonzeroparskip

% Huérfanas y viudas
\widowpenalty100000
\clubpenalty100000

% Imagenes
\usepackage{graphicx}
\newcommand{\imagen}[2]{
	\begin{figure}[!h]
		\centering
		\includegraphics[width=0.9\textwidth]{#1}
		\caption{#2}\label{fig:#1}
	\end{figure}
	\FloatBarrier
}

\newcommand{\imagenflotante}[2]{
	\begin{figure}%[!h]
		\centering
		\includegraphics[width=0.9\textwidth]{#1}
		\caption{#2}\label{fig:#1}
	\end{figure}
}



% El comando \figura nos permite insertar figuras comodamente, y utilizando
% siempre el mismo formato. Los parametros son:
% 1 -> Porcentaje del ancho de página que ocupará la figura (de 0 a 1)
% 2 --> Fichero de la imagen
% 3 --> Texto a pie de imagen
% 4 --> Etiqueta (label) para referencias
% 5 --> Opciones que queramos pasarle al \includegraphics
% 6 --> Opciones de posicionamiento a pasarle a \begin{figure}
\newcommand{\figuraConPosicion}[6]{%
  \setlength{\anchoFloat}{#1\textwidth}%
  \addtolength{\anchoFloat}{-4\fboxsep}%
  \setlength{\anchoFigura}{\anchoFloat}%
  \begin{figure}[#6]
    \begin{center}%
      \Ovalbox{%
        \begin{minipage}{\anchoFloat}%
          \begin{center}%
            \includegraphics[width=\anchoFigura,#5]{#2}%
            \caption{#3}%
            \label{#4}%
          \end{center}%
        \end{minipage}
      }%
    \end{center}%
  \end{figure}%
}

%
% Comando para incluir imágenes en formato apaisado (sin marco).
\newcommand{\figuraApaisadaSinMarco}[5]{%
  \begin{figure}%
    \begin{center}%
    \includegraphics[angle=90,height=#1\textheight,#5]{#2}%
    \caption{#3}%
    \label{#4}%
    \end{center}%
  \end{figure}%
}
% Para las tablas
\newcommand{\otoprule}{\midrule [\heavyrulewidth]}
%
% Nuevo comando para tablas pequeñas (menos de una página).
\newcommand{\tablaSmall}[5]{%
 \begin{table}
  \begin{center}
   \rowcolors {2}{gray!35}{}
   \begin{tabular}{#2}
    \toprule
    #4
    \otoprule
    #5
    \bottomrule
   \end{tabular}
   \caption{#1}
   \label{tabla:#3}
  \end{center}
 \end{table}
}

%
% Nuevo comando para tablas pequeñas (menos de una página).
\newcommand{\tablaSmallSinColores}[5]{%
 \begin{table}[H]
  \begin{center}
   \begin{tabular}{#2}
    \toprule
    #4
    \otoprule
    #5
    \bottomrule
   \end{tabular}
   \caption{#1}
   \label{tabla:#3}
  \end{center}
 \end{table}
}

\newcommand{\tablaApaisadaSmall}[5]{%
\begin{landscape}
  \begin{table}
   \begin{center}
    \rowcolors {2}{gray!35}{}
    \begin{tabular}{#2}
     \toprule
     #4
     \otoprule
     #5
     \bottomrule
    \end{tabular}
    \caption{#1}
    \label{tabla:#3}
   \end{center}
  \end{table}
\end{landscape}
}

%
% Nuevo comando para tablas grandes con cabecera y filas alternas coloreadas en gris.
\newcommand{\tabla}[6]{%
  \begin{center}
    \tablefirsthead{
      \toprule
      #5
      \otoprule
    }
    \tablehead{
      \multicolumn{#3}{l}{\small\sl continúa desde la página anterior}\\
      \toprule
      #5
      \otoprule
    }
    \tabletail{
      \hline
      \multicolumn{#3}{r}{\small\sl continúa en la página siguiente}\\
    }
    \tablelasttail{
      \hline
    }
    \bottomcaption{#1}
    \rowcolors {2}{gray!35}{}
    \begin{xtabular}{#2}
      #6
      \bottomrule
    \end{xtabular}
    \label{tabla:#4}
  \end{center}
}

%
% Nuevo comando para tablas grandes con cabecera.
\newcommand{\tablaSinColores}[6]{%
  \begin{center}
    \tablefirsthead{
      \toprule
      #5
      \otoprule
    }
    \tablehead{
      \multicolumn{#3}{l}{\small\sl continúa desde la página anterior}\\
      \toprule
      #5
      \otoprule
    }
    \tabletail{
      \hline
      \multicolumn{#3}{r}{\small\sl continúa en la página siguiente}\\
    }
    \tablelasttail{
      \hline
    }
    \bottomcaption{#1}
    \begin{xtabular}{#2}
      #6
      \bottomrule
    \end{xtabular}
    \label{tabla:#4}
  \end{center}
}

%
% Nuevo comando para tablas grandes sin cabecera.
\newcommand{\tablaSinCabecera}[5]{%
  \begin{center}
    \tablefirsthead{
      \toprule
    }
    \tablehead{
      \multicolumn{#3}{l}{\small\sl continúa desde la página anterior}\\
      \hline
    }
    \tabletail{
      \hline
      \multicolumn{#3}{r}{\small\sl continúa en la página siguiente}\\
    }
    \tablelasttail{
      \hline
    }
    \bottomcaption{#1}
  \begin{xtabular}{#2}
    #5
   \bottomrule
  \end{xtabular}
  \label{tabla:#4}
  \end{center}
}



\definecolor{cgoLight}{HTML}{EEEEEE}
\definecolor{cgoExtralight}{HTML}{FFFFFF}

%
% Nuevo comando para tablas grandes sin cabecera.
\newcommand{\tablaSinCabeceraConBandas}[5]{%
  \begin{center}
    \tablefirsthead{
      \toprule
    }
    \tablehead{
      \multicolumn{#3}{l}{\small\sl continúa desde la página anterior}\\
      \hline
    }
    \tabletail{
      \hline
      \multicolumn{#3}{r}{\small\sl continúa en la página siguiente}\\
    }
    \tablelasttail{
      \hline
    }
    \bottomcaption{#1}
    \rowcolors[]{1}{cgoExtralight}{cgoLight}

  \begin{xtabular}{#2}
    #5
   \bottomrule
  \end{xtabular}
  \label{tabla:#4}
  \end{center}
}


















\graphicspath{ {./img/} }

% Capítulos
\chapterstyle{bianchi}
\newcommand{\capitulo}[2]{
	\setcounter{chapter}{#1}
	\setcounter{section}{0}
	\chapter*{#2}
	\addcontentsline{toc}{chapter}{#2}
	\markboth{#2}{#2}
}

% Apéndices
\renewcommand{\appendixname}{Apéndice}
\renewcommand*\cftappendixname{\appendixname}

\newcommand{\apendice}[1]{
	%\renewcommand{\thechapter}{A}
	\chapter{#1}
}

\renewcommand*\cftappendixname{\appendixname\ }

% Formato de portada
\makeatletter
\usepackage{xcolor}
\newcommand{\tutor}[1]{\def\@tutor{#1}}
\newcommand{\course}[1]{\def\@course{#1}}
\definecolor{cpardoBox}{HTML}{E6E6FF}
\def\maketitle{
  \null
  \thispagestyle{empty}
  % Cabecera ----------------
\noindent\includegraphics[width=\textwidth]{cabecera}\vspace{1cm}%
  \vfill
  % Título proyecto y escudo informática ----------------
  \colorbox{cpardoBox}{%
    \begin{minipage}{.8\textwidth}
      \vspace{.5cm}\Large
      \begin{center}
      \textbf{TFG del Grado en Ingeniería Informática}\vspace{.6cm}\\
      \textbf{\LARGE\@title{}}
      \end{center}
      \vspace{.2cm}
    \end{minipage}

  }%
  \hfill\begin{minipage}{.20\textwidth}
    \includegraphics[width=\textwidth]{escudoInfor}
  \end{minipage}
  \vfill
  % Datos de alumno, curso y tutores ------------------
  \begin{center}%
  {%
    \noindent\LARGE
    Presentado por \@author{}\\ 
    en Universidad de Burgos --- \@date{}\\
    Tutor: \@tutor{}\\
  }%
  \end{center}%
  \null
  \cleardoublepage
  }
\makeatother

\newcommand{\nombre}{Pablo Fernández Bravo} %%% cambio de comando

% Datos de portada
\title{GTM - Plataforma de experimentación de Text Mining sobre repositorios de GitHub}
\author{\nombre}
\tutor{Dr. Carlos López Nozal y \\ D. Jesús María Alonso Abad}
\date{\today}

%in the preamble
%--------------------------------
  \usepackage[
    backend=biber,
    style=numeric,
  ]{biblatex}
  \usepackage{csquotes}

 \addbibresource{bibliografia.bib}
%--------------------------------

\begin{document}

\maketitle


\newpage\null\thispagestyle{empty}\newpage


%%%%%%%%%%%%%%%%%%%%%%%%%%%%%%%%%%%%%%%%%%%%%%%%%%%%%%%%%%%%%%%%%%%%%%%%%%%%%%%%%%%%%%%%
\thispagestyle{empty}


\noindent\includegraphics[width=\textwidth]{cabecera}\vspace{1cm}

\noindent Dr. Carlos López Nozal y D. Jesús María Alonso Abad, profesores del departamento de Ingeniería Informática, área de Lenguajes y Sistemas Informáticos.

\noindent Expone:

\noindent Que el alumno D. \nombre, con DNI 71364831-V, ha realizado el Trabajo final de Grado en Ingeniería Informática titulado <<GTM - Plataforma de experimentación de Text Mining sobre repositorios de GitHub>>. 

\noindent Y que dicho trabajo ha sido realizado por el alumno bajo la dirección del que suscribe, en virtud de lo cual se autoriza su presentación y defensa.

\begin{center} %\large
En Burgos, {\large \today}
\end{center}

\vfill\vfill\vfill

% Author and supervisor
\begin{minipage}{0.45\textwidth}
\begin{flushleft} %\large
Vº. Bº. del Tutor:\\[2cm]
Dr. Carlos López Nozal
\end{flushleft}
\end{minipage}
\hfill
\begin{minipage}{0.45\textwidth}
\begin{flushleft} %\large
Vº. Bº. del co-tutor:\\[2cm]
D. Jesús María Alonso Abad
\end{flushleft}
\end{minipage}
\hfill

\vfill

% para casos con solo un tutor comentar lo anterior
% y descomentar lo siguiente
%Vº. Bº. del Tutor:\\[2cm]
%D. nombre tutor


\newpage\null\thispagestyle{empty}\newpage




\frontmatter

% Abstract en castellano
\renewcommand*\abstractname{Resumen}
\begin{abstract}
Los sistemas de seguimiento de incidencias se han convertido en una herramienta esencial dentro de la gestión de un proyecto software. La naturaleza de este tipo de proyectos requiere de un importante mecanismo que se encargue de supervisar el progreso a lo largo de la resolución de los problemas. Este proceso de alto componente social requiere del establecimiento de un ecosistema que promueva la comunicación de conocimientos e ideas entre los miembros del equipo.

Este ecosistema presenta una oportunidad para la aplicación de tecnologías de \textbf{procesamiento del lenguaje natural (PLN)}, que a través de sus diferentes modelos proveen de mecanismos que tienen como objetivo el análisis y comprensión de las interacciones humanas con la tecnología. Estos modelos pueden colaborar en tareas como la clasificación de las incidencias, simplificando la comprensión de su finalidad y prioridad, la obtención de un extracto con aquella información más relevante dentro un reporte, o el análisis de la visión de los participantes con respecto una incidencia y la evolución de su resolución.

El objetivo de este proyecto consiste en el desarrollo de una plataforma que permita a los usuarios experimentar con la aplicación de las técnicas de PLN, anteriormente mencionadas a la gestión de tareas de aquellos proyectos de código abierto desarrollados a través de la plataforma \emph{GitHub}. Para simplificar el proceso al máximo, la plataforma será la encargada de extraer, almacenar, preprocesar, aplicar los modelos a la información recolectada, y mostrar los resultados obtenidos al usuario. Este alto nivel de automatización permitiendo al usuario centrarse en ajustar los parámetros para el lanzamiento de las distintas tareas y estudiar sus resultados.
\end{abstract}

\renewcommand*\abstractname{Descriptores}
\begin{abstract}
\emph{Análisis de Sentimientos}, aprendizaje supervisado, API REST, aplicación web, arquitectura de microservicios, \emph{código abierto}, experimentación, gestión de proyectos, inteligencia artificial, \emph{minería de textos}, Procesamiento del Lenguaje Natural, \emph{Resúmenes Abstractivos}, \emph{Clasificación Zero-Shot}, \emph{NLP}
\end{abstract}

\clearpage

% Abstract en inglés
\renewcommand*\abstractname{Abstract}
\begin{abstract}
Issue tracking systems have become an essential tool in software project management. The nature of this type of project requires an important mechanism to monitor progress throughout the resolution of the issues. This highly social process requires the establishment of an ecosystem that promotes the communication of knowledge and ideas between team members.

This ecosystem presents an opportunity for the application of \textbf{natural language processing (NLP)} technologies, which through their different models provide mechanisms that aim to analyze and understand human interactions with technology. These models can assist in tasks such as issue classification, simplifying the understanding of their purpose and priority, obtaining an extract with the most relevant information within a report, or analyzing the vision of the participants with respect to an incident and the evolution of its resolution.

The objective of this project is to develop a platform that allows users to experiment with the application of the NLP techniques mentioned above to the task management of those open source projects developed through the \emph{GitHub} platform. To simplify the process as much as possible, the platform will be in charge of extracting, storing, preprocessing, applying the models to the gathered information and displaying the achieved results to the user. This high level of automation allows the user to focus on adjusting the parameters for running the different tasks and studying their results.
\end{abstract}

\renewcommand*\abstractname{Keywords}
\begin{abstract}
Sentiment Analysis, supervised learning, REST API, web application, microservices architecture, open source, experimentation, project management, artificial intelligence, text mining, Natural Language Processing, Summarization, Zero-Shot Classification, NLP
\end{abstract}

\clearpage

% Indices
\tableofcontents

\clearpage

\listoffigures

\clearpage

% \listoftables
% \clearpage

\mainmatter
\capitulo{1}{Introducción}

Los proyectos software requieren del establecimiento de una serie de medidas que permitan organizar y gestionar su progreso a lo largo de las diversas fases del desarrollo. Entre estas medidas en los proyectos software destacan los \textbf{sistemas de gestión y seguimiento de incidencias}. La tendencia en el uso de metodologías rápidas en este tipo de proyectos implica la necesidad de dividir los grandes proyectos en cortas iteraciones compuestas por pequeñas tareas. La asignación de este tipo de tareas suele llevarse a cabo de manera individual, debido a que de manera diaria se realizan reuniones donde exponer los problemas encontrados y proponer soluciones e ideas en conjunto.

La organización de este tipo de reuniones no siempre tiene porqué llevarse a cabo de manera presencial, es posible realizarse a través de foros internos donde debatir las propuestas, clasificar las tareas, gestionar las incidencias detectadas, y en definitiva analizar la situación de las pequeñas tareas con el objetivo de concluir el proyecto de acuerdo con los requisitos en tiempo y recursos establecidos. En este punto es donde surgen programas o plataformas con la finalidad de simplificar la gestión de las tareas en grandes proyectos, cuya organización puede resultar compleja y caótica.

El inconveniente de este tipo de plataformas recae en que pese a facilitar las labores de seguimiento del estado de los procesos o incidencias, siguen requiriendo un alto grado de atención manual por parte del encargado de su gestión. La razón detrás de ello recae en el hecho de que la definición, descripción, debate, y conclusión de las tareas poseen un alto componente social que no podía ser interpretado por las máquinas.

El punto de inflexión con los problemas anteriores llega en los últimos 10 años, momento en el cual las empresas comenzaron a descubrir el potencial de la inmensa cantidad de información que estas habían almacenado durante décadas. La necesidad de llevar a cabo un tratamiento de estos datos propició el creciente desarrollo y evolución de las \textbf{técnicas de procesamiento del lenguaje natural (PLN)}. Estas técnicas tienen como principal objetivo la aplicación de \textbf{Inteligencia Artificial (IA)} para el análisis automático de las comunicaciones humanas por medio de los computadores, permitiendo la extracción, clasificación y descubrimiento de detalles e información relevante que no aparece dispuesta de manera explícita.

Estas técnicas de procesamiento del lenguaje natural en conjunto con la necesidad de automatizar aquellas labores de seguimiento de incidencias en la gestión de un proyecto es el lugar donde se sitúa nuestro objetivo con este trabajo. Con la finalidad de comprobar y demostrar el potencial de esta especialidad del aprendizaje automático en su aplicación a las labores de seguimiento de incidencias de un proyecto software este trabajo plantea el diseño de una plataforma con la que experimentar con distintos modelos de PLN aplicables y su ajuste a través de la modificación de parámetros.  

Los modelos de PLN que se ha decidido incluir en la plataforma permiten al usuario conocer la probabilidad con la que el contenido de una incidencia se ajusta a una serie de etiquetas definidas tanto en el propio repositorio como introducidas por el usuario (\textbf{\emph{Zero-Shot Classification}}), generar resúmenes abstractivos que sinteticen el contenido de las incidencias (\textbf{\emph{Summarization}}), y obtener gráficos que señale la postura de los participantes con respecto a una incidencia deducida a partir del análisis de su actitud en los comentarios (\textbf{\emph{Sentiment Analysis}}).

\capitulo{2}{Objetivos del proyecto}

Este capítulo tiene como objeto el recoger los objetivos generales que se establecieron al comienzo del proyecto con la finalidad de determinar y acotar el alcance del proyecto, así como los objetivos técnicos con el propósito de determinar las técnicas y tecnologías clave que se acordó utilizar.

\section{Objetivos generales}

En este apartado se exponen los objetivos de carácter general que se pactó implementar en el proyecto. A continuación se describen estos objetivos:

\begin{itemize} [\textbullet]

\item Realizar una aplicación web que permita al usuario experimentar con múltiples técnicas de procesamiento del lenguaje natural (PLN). Tras estudiar los posibles métodos que aplicar se decidió que en esta ocasión la plataforma trabajará con tareas de \emph{clasificación Zero-Shot}, el análisis de sentimientos y la generación de resúmenes abstractivos.

\item Los modelos que se van a utilizar no se implementarán y entrenarán desde cero, en su lugar se recurrirá a la explotación del potencial de los \textbf{modelos preentrenados}. Mediante portales como \emph{Hugging Face} \cite{platform:hugging_face} es posible obtener modelos que aplican en multitud de campos con el beneficio de haber sido entrenados y ajustados por anterioridad. Este tipo de iniciativas realza el valor del software de código abierto y permiten que pequeños proyectos dispongan de acceso a potentes herramientas que ofrecen resultados con suficiente grado de calidad.

\item Los datos sobre los que se aplicarán los modelos deberán ser extraídos desde las incidencias de repositorios de código abierto de \emph{GitHub} \cite{platform:github}. La motivación de que los datos de entrada se obtengan mediante esta vía surge con el fin de demostrar los beneficios de aplicar este tipo de metodologías a la gestión de tareas dentro de un proyecto. A partir de la aplicación se podría realizar una clasificación automática o semiautomática de tareas, obtener un análisis del estado moral del equipo u ofrecer una sinopsis de las tareas y su estado.

\item Enfocar la \hyperref[sec:arquitectura]{arquitectura y diseño del proyecto} teniendo en cuenta la posibilidad de aplicar mejoras sobre el mismo y ampliar sus funcionalidades de cara futuras iteraciones tanto en el \emph{front-end} como en el \emph{back-end}. Esto incluye dotar al código de una buena documentación que permita a futuros desarrolladores comprender el funcionamiento interno de la aplicación.
\item Trasladar a la práctica aquellos conocimientos adquiridos a lo largo del grado en las materias de Ingeniería del Software, Gestión de Proyectos, Metodología de la Programación, Bases de Datos, Algoritmia, Diseño y Mantenimiento del Software, y Minería de Datos.
\item Ofrecer el código bajo una licencia que posibilite su libre uso y acceso, permitiendo la continuidad del proyecto más allá del alcance de este trabajo.

\end{itemize}

\bigskip

\section{Objetivos técnicos}

Este apartado tiene como finalidad presentar aquellos desafíos técnicos que afrontar de cara a dotar al proyecto de la tecnología necesaria para la implementación de sus funciones. Estos objetivos mencionados son los siguientes:

\begin{itemize} [\textbullet]

\item El proyecto debe plantearse teniendo en mente una \hyperref[sec:microservicios]{\textbf{arquitectura de microservicios}} que otorgue de independencia a cada fase del proceso. El objetivo de estar arquitectura consiste en otorgar la libertad de que en caso de que se requiriese una ampliación o reubicación de recursos por parte de uno de los servicios, estos puedan ser ajustados de manera individual sin alterar el funcionamiento completo de la aplicación. 

\item Se deberá contar con un servicio encargado de establecer la conexión con la \emph{API} de \emph{GitHub} \cite{platform:github_api} para la \hyperref[sec:extraccion]{extracción de los datos} Este servicio a su vez se encargará de extraer aquella información que así se requiera de las incidencias, aplicar un preprocesado sobre estos datos, y una vez se obtengan los datos en limpio, se proceda a su almacenamiento en la base de datos.

\item La base de datos que se escoja deberá soportar los accesos de manera concurrente por parte de los múltiples servicios que compondrán la aplicación. Esta se desplegará en un servicio independiente para facilitar su mantenimiento.

\item Los \hyperref[sec:preentrenados]{\textbf{modelos preentrenados}} ofrecen la posibilidad de modificar ciertos parámetros con el objetivo de adaptar los resultados a la salida deseada. Siempre que sea posible se ofrecerá al usuario de la aplicación la posibilidad de configurar estos parámetros de acuerdo con sus preferencias para dotar a la aplicación de una mayor funcionalidad.

\item Con el fin de mejorar los resultados de los \hyperref[sec:preentrenados]{\textbf{modelos preentrenados}} se estudiarán las limitaciones en el tamaño de sus entradas y se implementarán técnicas de \hyperref[sec:preprocesamiento]{preprocesado} que permitan mitigar sus efectos en situaciones donde los textos sean demasiado grandes.

\end{itemize}

\capitulo{3}{Conceptos teóricos}

Este capítulo tiene como objetivo exponer de manera detallada los conceptos teóricos que consolidan las bases del proyecto. Para facilitar la comprensión de los numerosos conceptos que han hecho su aparición en el desarrollo se ha tomado la decisión de agrupar estos en 3 secciones. Primeramente, se desarrollarán aquellos \hyperref[sec:gestionproyectos]{conceptos relacionados con la organización y gestión de proyectos}. Seguidamente, se expondrán \hyperref[sec:conceptosextraccionprocesamiento]{nociones relativas a la extracción y el preprocesamiento de los datos}. Finalmente, se presentarán aquellos \hyperref[sec:procesamientodatos]{conceptos en lo referente a los modelos de procesamiento del lenguaje natural (PLN)}.

\section{Conceptos teóricos relativos a la gestión de proyectos} \label{sec:conceptosgestionproyectos}

Uno de los principales conceptos con los que se presentó la idea original de proyecto radicaba en trabajar con la idea de la \hyperref[sec:gestionproyectos]{gestión de proyectos} aplicando técnicas de \textbf{Inteligencia Artificial (IA)}. La gestión de proyectos es una disciplina de la administración de empresas encargada de la aplicación de metodologías de planificación, organización, motivación y control de recursos con el objetivo de obtener un resultado único.

Como podemos deducir de la anterior definición, la gestión de proyectos está compuesta por un gran número de disciplinas que interaccionan entre sí para conducir un proyecto a un claro objetivo propuesto. El éxito de un proyecto se computa recurriendo al estudio del cumplimiento de los requisitos de este, así como las relaciones entre el tiempo, el coste y el alcance de este.

De acuerdo con el informe <<\emph{CHAOS 2020: Beyond Infinity Overview}>> \cite{ct:portman_project_success} a fecha de enero de 2021 tan solo un \textbf{31\%} de los proyectos resultan exitosos. Estos datos se presentan frente a un \textbf{50\%} que se presentan como proyectos que aun habiendo cumplido su objetivo no lograron efectuarlo respetando las restricciones inicialmente establecidas.

\subsection{Gestión de Proyectos (\emph{Project Management})} \label{sec:gestionproyectos}

Uno de los principales desafíos a los que se enfrentan los proyectos se encuentra en el proceso de planificación, gestión y seguimiento de las tareas de manera que se cumpla con los requisitos y objetivos propuestos.  \cite{ct:universal_knowledge_management}

La gestión de tareas consiste en el proceso de llevar a cabo el seguimiento de una tarea a lo largo de su ciclo de vida. Este proceso se compone de compone de una serie de fases de planificación, pruebas, \hyperref[sec:seguimientoincidencias]{seguimiento} (tracking) y generación de informes. El objetivo de la gestión de tareas es lograr que cada tarea alcance su objetivo cumpliendo con los requisitos establecidos para cada tarea individual, así como lograr establecer una comunicación entre los responsables de llevar a cabo cada tarea de manera que se consigan completar los objetivos globales.

\subsection{Seguimiento de Incidencias (\emph{Issue Tracking})} \label{sec:seguimientoincidencias}

El seguimiento de incidencias consiste en la aplicación de una serie de metodologías cuya finalidad es cumplir con la planificación inicial establecida para un proyecto manteniendo un control sobre las tareas que lo componen.

La importancia del \emph{issue tracking} recae en que los problemas que se generan en una tarea repercuten en la planificación de tareas posteriores, y en última instancia en la planificación general del proyecto, es por ello por lo que es de vital importancia localizar agilizar localización de las dificultades en el desarrollo de las tareas para que estas puedan ser solventadas con la mayor brevedad y de la manera más eficiente posible.

Para facilitar el seguimiento de un proyecto se plantea el uso de herramientas conocidas como \emph{project trackers}. Este tipo de software permite planificar el desarrollo de un proyecto estableciendo metas parciales, creando objetivos medibles que se ajusten los requisitos de manera realista, llevando un registro de los recursos o personal implicado en cada etapa o tarea del proyecto, designando periodos de reunión que fortalezcan la comunicación entre los diferentes participantes del proyecto compartiendo avances y propuestas, y estableciendo márgenes de maniobra con el objetivo de mitigar los posibles efectos de los problemas que aparezcan a lo largo de su desarrollo.

\subsection{Plataformas (\emph{GitHub})} \label{sec:plataformas}

Existen numerosas plataformas surgidas en base a la necesidad de gestionar el desarrollo de un proyecto. Por lo general, estas plataformas ofrecen soluciones que permiten almacenar y llevar un control de versiones del proyecto, y a su vez ofrecer sistemas de gestión y seguimiento de las tareas. La ventaja de esta unificación supone el poder mantener en una misma situación los detalles de los cambios, vincularlos a las incidencias y mantener las discusiones sobre su desarrollo e implementación en los comentarios de la tarea.

En el desarrollo de nuestra aplicación hemos centrado los esfuerzos en aplicar el proceso de extracción y experimentación sobre proyectos de la plataforma \emph{GitHub}. La motivación principal detrás de escoger esta plataforma, frente a sus principales alternativas \emph{GitLab} y \emph{BitBucket}, es el número de proyectos que se desarrollan a través de \emph{GitHub}.

En las siguientes secciones se van a presentar brevemente los tres elementos principales de \emph{GitHub} en los que se ha centrado la atención del proyecto: \hyperref[sec:repositorio]{los repositorios}, \hyperref[sec:incidencia]{las incidencias}, y \hyperref[sec:etiqueta]{las etiquetas}.

\subsubsection{Repositorio} \label{sec:repositorio}

Un repositorio \cite{ct:github_repository} es el concepto principal alrededor del que se construye la plataforma \emph{GitHub}. Cada repositorio representa el desarrollo de un proyecto, generalmente software. Estos repositorios permiten como funcionalidad principal la gestión y el almacenamiento de su contenido a través de un sistema control de versiones basado en la tecnología \emph{Git}. Este planteamiento permite a los participantes del proyecto poder trabajar de manera simultánea en distintas tareas y a su vez la posibilidad mantener el desarrollo siempre actualizado con los últimos cambios disponibles.

Entendiendo los repositorios como instancias de alto nivel dentro de la plataforma, estos a su vez están compuestos por multitud de entidades como son los propios ficheros que componen el proyecto, el historial de versiones de estos, las incidencias, las peticiones de cambios o los hitos. En este trabajo nos centraremos en aquella información relacionada de manera estricta con las incidencias.

\subsubsection{Incidencia} \label{sec:incidencia}

Una incidencia \cite{ct:github_issue} de un repositorio \emph{GitHub} simboliza una tarea pendiente o resuelta dentro de un proyecto. Estas tareas pueden representar peticiones de cambios, solicitudes nuevas características, o informes de errores. Estas incidencias pueden ser creadas por aquellas personas con acceso al repositorio en cuestión, las cuales en el caso de los repositorios de código abierto son cualquier persona que disponga de una cuenta en la plataforma.

\begin{figure}[!ht]
	\centering
	\includegraphics[width=\textwidth]{img/gh_repository_issue.png}
	\caption{Aspecto de una incidencia en GitHub.}
	\label{fig:gh_repository_issue}
\end{figure}

Las incidencias disponen de múltiples propiedades mediante las cuales detallar su finalidad, establecer su clasificación con respecto a las temáticas definidas en forma de etiquetas por el repositorio, asignar su resolución a personas concretas implicadas en el desarrollo del proyecto, gestionar su resolución y progreso en torno a hitos o versiones, comentarios realizados por el creador u otros usuarios que permiten complementar, extender o proponer resoluciones y mejoras a lo planteado inicialmente.

\subsubsection{Etiqueta} \label{sec:etiqueta}

En \emph{GitHub} una etiqueta \cite{ct:github_label} es elemento conformado por un identificador, un código de color hexadecimal y opcionalmente, una descripción que explique más en detalle la intención que señala dicho identificador. Las etiquetas son declaradas a nivel de repositorio, y pueden ser asignadas a incidencias, peticiones de cambios y discusiones. La intención de estas etiquetas consiste en establecer una clasificación de los diferentes objetos del repositorio entorno a una serie de categorías que permitan localizar de manera más rápida aquellos elementos relacionados con la temática de la etiqueta.

Por defecto en la creación de un repositorio se añaden una serie de etiquetas que posteriormente pueden ser editadas o eliminadas. Estas etiquetas base son: \emph{bug}, \emph{documentation}, \emph{duplicate}, \emph{enhancement}, \emph{good first issue}, \emph{help wanted}, \emph{invalid}, \emph{question} y \emph{wontfix} \cite{ct:github_managing_labels}.

\begin{figure}[!ht]
	\centering
\includegraphics[width=\textwidth]{img/gh_repository_tags.png}
	\caption{Etiquetas de un repositorio en GitHub.}
	\label{fig:gh_repository_tags}
\end{figure}

\section{Conceptos teóricos relativos a la extracción y procesamiento de los datos} \label{sec:conceptosextraccionprocesamiento}

\subsection{Minería de Datos} \label{sec:mineriadatos}

La minería de datos consiste en la aplicación de técnicas de \textbf{Inteligencia Artificial (IA)} sobre grandes cantidades de datos cuyo objetivo trata de localizar patrones, tendencias o establecer relaciones entre estos. Dentro de las técnicas que componen la IA, la minería de datos se relaciona con campos de aprendizaje como el aprendizaje computacional, la estadística y las bases de datos. Este conjunto de metodologías se engloba dentro de una etapa denominada \textbf{Descubrimiento de Conocimiento en Bases de Datos (KDD)}, la cual tiene como objetivo someter a los datos a una serie de restricciones de eficiencia que dan lugar a una enumeración de los patrones localizados \cite{ct:machine_learning}.

La minería de datos permite la aplicación de multitud de técnicas para el descubrimiento de patrones en los datos. A modo de establecer una separación de acuerdo con el tipo de patrones que tratan de descubrir al aplicar dichas técnicas surgen las siguientes tres categorías principales: \hyperref[sec:mddprediccion]{predicción}, \hyperref[sec:mddanalisisasociaciones]{análisis de asociaciones} y \hyperref[sec:mddclustering]{clustering}.

\subsubsection{Predicción} \label{sec:mddprediccion}

La predicción engloba aquellas tareas que tienen como objetivo establecer relaciones entre el vector de características que conforman los elementos a clasificar y el atributo objetivo con el que se desea determinar su relación. En las tareas de predicción podemos señalar dos categorías en función del tipo de atributo objetivo que se esté prediciendo:

\vspace{-0.5cm}
\begin{itemize} [\textbullet]
	\item \textbf{Clasificación}. Las tareas de clasificación se caracterizan debido a que el atributo objetivo debe pertenecer a una clase determinada de entre un número limitado de opciones establecidas al comienzo del experimento.
	\item \textbf{Regresión}. En las tareas de regresión su principal característica se encuentra en el hecho de que el atributo objetivo debe ser de tipo numérico y su valor comprenderá un intervalo infinito de valores.
\end{itemize}

\subsubsection{Análisis de asociaciones} \label{sec:mddanalisisasociaciones}

En el análisis de asociaciones el objetivo es la revelación de hechos que suceden de manera conjunta dentro de un conjunto determinado de datos. Estos datos revelados pueden retornar cualquier atributo de los elementos implicados e incluso combinaciones entre ellos.

\subsubsection{Clustering} \label{sec:mddclustering}

El clustering comprende la aplicación de técnicas de machine learning y aprendizaje supervisado con el objetivo de identificar subconjuntos de datos con similares características entre los elementos que los componen dentro de un gran conjunto de datos.

\subsection{Recogida de los datos} \label{sec:extraccion}

La recogida de los datos la primera fase dentro de cualquier proyecto de minería de datos. Es necesario realizar un estudio previo acotando los datos que se desea conseguir y la forma en la que se va a llevar a cabo su recolección.

\begin{figure}[!ht]
	\centering
\includegraphics[width=\textwidth]{img/extraction_process.png}
	\caption{Esquema que representa el proceso seguido para la extracción de la información desde un repositorio GitHub a partir de una petición generada desde la plataforma.}
	\label{fig:extraction_process}
\end{figure}

En nuestro proyecto la extracción de los datos requiere del lanzamiento de peticiones contra la API REST de GitHub. El ámbito de la información a recoger contempla la dirección, título, descripción y etiquetas del repositorio, así como el título, autor, cuerpo, etiquetas y comentarios de las incidencias que este posea.

\subsection{Preprocesamiento de los datos} \label{sec:preprocesamiento}

Los datos en crudo extraídos por medio de la conexión con GitHub poseen impurezas que deben ser filtradas con el objetivo de almacenar la información estrictamente necesaria para la posterior aplicación de los diferentes modelos de procesamiento. Estas impurezas pueden aparecer principalmente debido al lenguaje de marcado Markdown, el cual es soportado por GitHub en el cuerpo de las incidencias y comentarios de los repositorios. Su uso está muy extendido a lo largo de los usuarios de GitHub ya que mediante la incorporación de secuencias fáciles de recordar permiten estilizar los textos de modo que al renderizarse la información facilita su lectura.

\begin{figure}[!ht]
	\centering
\includegraphics[width=\textwidth]{img/extraction_preprocessing_process.png}
	\caption{Esquema que representa los procesos de preprocesado a los que se somete la información extraída desde GitHub.}
	\label{fig:apply_preprocessing}
\end{figure}

Otras impurezas con las que podemos encontrarnos es con la aparición de fragmentos de código entre medias de los textos. Esta información resulta complementen irrelevante para el propósito que nos concierne y serán eliminados para evitar confundir a los modelos.

\section{Conceptos teóricos relativos al Procesamiento de los Datos} \label{sec:procesamientodatos}

\subsection{Aprendizaje profundo (Deep learning)} \label{sec:deeplearning}

El aprendizaje profundo \cite{ct:deep_learning} consiste en un conjunto de metodologías de aprendizaje automático basadas en la aplicación de redes neuronales artificiales que hacen uso del aprendizaje por representación. Este último tipo de aprendizaje por representación otorga al aprendizaje profundo una de sus principales propiedades, la simulación del comportamiento del cerebro humano. Para llevar a cabo esta compleja labor se requiere del uso de redes neuronales artificiales compuestas por multitud de capas de neuronas interconectadas, y es por ello este aprendizaje recibe el calificativo de “aprendizaje profundo”.

El proceso de entrenamiento que se sigue para este tipo de técnicas puede variar entre aprendizaje supervisado \cite{wiki:supervised_learning}, aprendizaje parcialmente supervisado \cite{wiki:semi_supervised_learning} o aprendizaje no supervisado \cite{wiki:unsupervised_learning} en función de las entradas que se dispongan y el tipo de tareas para las que se haya diseñado.

\subsection{Procesamiento del lenguaje natural (Natural lenguage processing)} \label{sec:pln}

El Procesamiento del Lenguaje Natural (PLN) \cite{ct:pln_universidad_san_marcos} es un subconjunto de técnicas de aprendizaje profundo cuyos métodos tratan de generar modelos de inteligencia artifical (IA) enfocados en el estudio y comprensión de los mecanismos que conforman el lenguaje humano de manera que posteriormente sean capaces de reproducir estos comportamientos. En los últimos años los sistemas basados en el PLN han adquirido una importante popularidad debido a la amplia variedad de tareas cuyos comportamientos son aplicables. Entre sus aplicaciones más destacables se encuentran: la \hyperref[sec:summarization]{generación de resúmenes}, la \hyperref[sec:clasificadorzeroshot]{clasificación de textos}, la traducción, el \hyperref[sec:analisissentimientos]{análisis de sentimientos} y los agentes conversacionales.

\subsection{Modelos preentrenados aplicados al PLN} \label{sec:preentrenados}

Los modelos de PLN requieren de un exhaustivo entrenamiento previo a su puesta en marcha con el objetivo de lograr obtener unos buenos resultados. Para ello se precisa de grandes conjuntos de datos previamente depurados para permitir al modelo establecer relaciones entre los datos y asimilar los patrones localizados.

\begin{figure}[!ht]
	\centering
\includegraphics[width=\textwidth]{img/applying_nlp_process.png}
	\caption{Esquema que representa el proceso seguido para aplicación de los modelos de PLN a partir de una petición generada desde la plataforma.}
	\label{fig:apply_nlp_process}
\end{figure}

Los modelos preentrenados \cite{ct:pretrained_models} son una nueva tendencia que trata de minimizar los recursos invertidos en la construcción desde cero de este tipo de modelos. Su fundamento consiste en proporcionar modelos de propósito general entrenados a partir de grandes \emph{datasets} con la premisa de proporcionar una base funcional a los desarrolladores de este tipo de proyectos. El inconveniente de este tipo de modelos se encuentra en la pérdida de precisión en campos de aplicación específicos, aunque estos efectos siempre pueden mitigarse a través de la introducción de ajustes en los parámetros de configuración del modelo.

\subsubsection{Zero Shot Classification} \label{sec:clasificadorzeroshot}

El modelo preentrenado <<\textbf{bart-large-mnli}>> \cite{huggingface:bart_large_mnli} diseñado por la división de Inteligencia Artificial de \emph{Facebook} es el escogido para la aplicación del procesado utilizando la técnica de \emph{Zero-Shot Classification}. Este modelo ha sido entrenado haciendo uso del \emph{dataset} \textbf{MultiNLI} y está basado en el modelo preentrenado \textbf{Bidirectional and Auto-Regressive Transformer (BART)}.

El modelo se basa en la tecnología de modelos \textbf{Inferencia del Lenguaje Natural (NLI)}, un subconjunto de los modelos de Procesamiento del Lenguaje Natural (PLN), los cuales tratan de determinar si una hipótesis es verdadera, falsa o neutral a partir de una premisa dada. Estos sistemas son entrenados a partir de grandes \emph{datasets} que contienen una serie de premisas, una hipótesis por cada premisa y una etiqueta que identifica la relación entre ambas.

El funcionamiento del concreto de este modelo consiste en la \textbf{generación de hipótesis} a partir de la secuencia a analizar y cada una de las etiquetas introducida. Una vez se construyen las hipótesis estas son evaluadas de acuerdo con sus \textbf{índices de implicación y contradicción} obteniendo un valor para cada hipótesis. Estos valores serán los asignados a las etiquetas como probabilidad de que exista una relación entre la cadena introducida y la etiqueta.

\subsubsection{Sentiment Analysis} \label{sec:analisissentimientos}

La elección de una implementación para la aplicación de \emph{Sentiment Analysis} \cite{huggingface:bert_base_multilingual_uncased_sentiment}es el modelo preentrenado <<\textbf{bert-base-multilingual-uncased-sentiment}>> desarrollado por \emph{NLP Town} partiendo como base del modelo \textbf{Bidirectional Encoder Representations from Transformers (BERT)}. Esta implementación en concreto ha sido entrenada con más de 620k reseñas de productos en seis idiomas distintos: inglés, francés, alemán, holandés, italiano y español.

Su funcionamiento trata de predecir el \textbf{grado de satisfacción} de la secuencia introducida otorgándole una puntuación \textbf{entre 1 y 5 estrellas}. Según las pruebas a las que ha sido sometido este modelo, la precisión con la que es capaz de predecir la puntuación otorgada de manera exacta es de media un \textbf{60\%}, que se amplía hasta un \textbf{94\%} en el caso de predicciones con una desviación de un nivel por encima o por debajo de la valoración original. Además de retornar la puntuación en estrellas, este modelo es capaz de retornar la puntuación exacta calculada, que es el valor que utilizaremos en nuestro caso para reflejar los resultados obtenidos.

\subsubsection{Summarization} \label{sec:summarization}

En el caso de la tarea de \emph{Summarization} \cite{huggingface:distilbart_cnn_12_6} el modelo preentrenado utilizado es el <<\textbf{distilbart-cnn-12-6}>> creado por \emph{Sshleifer}. Este modelo se basa en una versión reducida del modelo \emph{BART} denominada \textbf{DistilBART} que reduce los tiempos de generación a costa de una pequeña pérdida de rendimiento.

Existen dos tipos de técnicas utilizadas en la generación de resúmenes:

\begin{itemize} [\textbullet]
    \item \textbf{Generación de resúmenes por extracción}. La generación de resúmenes extractivos tiene como objetivo la obtención de resúmenes a partir de \textbf{fragmentos extraídos del texto original}. Para llevar a cabo este proceso, el modelo se encarga de puntuar las diferentes oraciones que componen el enunciado introducido de acuerdo a una predicción de la importancia que tiene su contenido sobre el global de la información original. A partir de este \emph{ranking} el modelo escoge los enunciados de mayor puntuación con la finalidad de que el texto obtenido condense las principales ideas del enunciado original.
    
    \item \textbf{Generación de resúmenes por abstracción}. La generación de resúmenes por medio de técnicas de abstracción tiene como objetivo la generación de \textbf{nuevo contenido} a partir de las ideas del texto original. El proceso de generación de este tipo de resúmenes inicialmente se encarga de mezclar las oraciones del enunciado introducido. El siguiente paso consisten en reemplazar ciertos fragmentos del texto por máscaras. Por último, estas máscaras son reemplazadas por un pequeño conjunto de palabras, una palabra o un símbolo de puntuación.
\end{itemize}

El modelo escogido tiene como propósito la generación de resúmenes abstractivos adecuándose a los parámetros de longitud mínima y máxima establecidos. El entrenamiento del modelo fue llevado a cabo haciendo uso del \emph{dataset} de \textbf{cnn\_dailymail} \cite{ct:dailymail}, el cual contiene numerosos artículos periodísticos que son tratados como entrada, y una serie de ideas clave del artículo que se unen formando el texto objetivo que se pretende alcanzar.

\capitulo{4}{Técnicas y herramientas} \label{chapter:tecnicas_y_herramientas}

Este capítulo tiene la finalidad de resaltar aquellas tecnologías, implementaciones de software de terceros y otro tipo de herramientas más relevantes que han sido utilizadas a lo largo del desarrollo del proyecto. Con el objetivo de organizar las tecnologías implicadas, estas han sido organizadas en torno a tres secciones:

\vspace{-0.35cm}
\begin{itemize} [\textbullet]
	\item \hyperref[sec:tecnologias]{\textbf{Tecnologías}}. Esta sección pretende reflejar información relevante acerca de los diferentes lenguajes y tecnologías que han sido utilizadas a lo largo del desarrollo.
	\item \hyperref[sec:dependencias]{\textbf{Dependencias}}. En esta sección se pretende indicar aquellos complementos más relevantes utilizados para ampliar las funcionalidades de \emph{Python} y \emph{NodeJS}.
	\item \hyperref[sec:herramientas]{\textbf{Herramientas}}. Esta sección recopila las herramientas utilizadas para el desarrollo y documentación del proyecto.
\end{itemize}

\section{Tecnologías} \label{sec:tecnologias}

\subsection{Python}

\textbf{Python} \cite{lang:python} es un lenguaje de programación de código abierto, interpretado, y orientado a objetos con soporte de programación estructurada por procedimientos y programación funcional administrado por la Python Software Foundation \cite{tech:python_software_fundation}. Entre sus características más destacadas se encuentran la posibilidad de incorporar módulos que extiendan las funcionalidades del lenguaje de manera rápida y sencilla a través de su gestor de paquetes “PIP”, el tipado dinámico, la gestión automática de la memoria a través de su recolector de basura \cite{wiki:garbage_collection}, la amplia cobertura de funcionalidades que ofrece su biblioteca estándar y el alto grado de legibilidad que provee por medio de un estricto conjunto de reglas de indentación \cite{tech:advanced_python}.

El uso de Python en nuestro proyecto resultará de vital importancia ya que es el lenguaje escogido como base para el desarrollo e implementación de los \hyperref[sec:microservicios]{microservicios}. En adición a sus bibliotecas estándares, cada módulo implementará una serie de paquetes que dotarán a estos de funcionalidades adicionales que faciliten la resolución de problemas para los que ya se posee una implementación previa que se ajuste a las necesidades del proyecto.

Su elección frente a otro tipo de lenguajes back-end, como podrían ser Java o JavaScript con NodeJS, viene definida a razón de la experiencia previa con dicho lenguaje en la construcción de arquitecturas basadas en microservicios a lo largo de la asignatura de Diseño y Mantenimiento del Software.


\subsection{JavaScript}

\textbf{JavaScript (JS)} \cite{lang:javascript} es un lenguaje de programación interpretado que implementa el estándar \emph{ECMAScript} \cite{tech:ecmascript}. Entre sus principales características se encuentran su orientación a objetos por medio de programación basada en prototipos, el ser un lenguaje débilmente tipado, poseer tipado dinámico y el contar con funciones de primera clase \cite{tech:first_class_func}.

Entre las principales aplicaciones del lenguaje se encuentra su uso como lenguaje de scripting en aplicaciones web, aunque debido a su versatilidad goza de un amplio uso en el lado del servidor a través de entornos como \emph{NodeJS}. La manera mediante la cual interactúa con la web se debe al amplio soporte del que goza por parte de los navegadores modelos, todos ellos integrando un intérprete de JavaScript, y el acceso que estos le otorgan a los elementos que componen la web a través del \emph{Document Object Model (DOM)} \cite{tech:dom}.

No es necesario presentar grandes razones para conocer la decisión de escoger JavaScript como lenguaje de programación en el apartado front-end de la aplicación. JavaScript es el lenguaje más popular de desarrollo web, posee un amplio soporte en los principales navegadores web, se dispone de multitud de recursos físicos, digitales, textuales y audiovisuales para comprender su funcionamiento y posee una amplia comunidad de bibliotecas y frameworks que facilitan la ampliación de sus capacidades.

\subsection{ReactJS}

ReactJS \cite{tech:react} es una biblioteca para JavaScript de código abierto multiplataforma, mantenida por Facebook y aquellos desarrolladores que libremente hayan contribuido al proyecto, cuyo objetivo principal es facilitar la creación y diseño de interfaces de usuario a través del uso de aplicaciones en una sola página (SPA). React tiene la finalidad de actuar como la vista dentro de la arquitectura Modelo-Vista-Controlador (MVC) frente a otras alternativas, aunque su amplio ecosistema de módulos posibilita la resolver de otro tipo de cuestiones como el enrutamiento.

Una de las características principales de React es su Virtual DOM \cite{tech:react_dom}, el cual se encarga de lidiar de manera transparente y eficiente con la actualización del contenido del DOM. Esto se logra debido a que en el momento que React detecta un cambio en la vista, este almacena estos cambios en su DOM Virtual, que es mucho más rápido que realizar una actualización completa del DOM del navegador. A partir de estos datos almacenados React lleva a cabo una comparación entre el estado previo, que se encuentra almacenado en su Virtual DOM, y el estado actual del DOM, detectando cuales son los componentes de la web que han sido modificados y deben de ser actualizados de manera individual, y cuales deben permanecen en el mismo estado.

La elección de la biblioteca de React frente a otros frameworks de propósito similar como VueJS o Angular no fue sencilla ya que todos ellos poseen características que resultan muy convenientes a la hora de trabajar con ellos. Finalmente, escoger esta biblioteca vino determinado por el nivel de madurez del proyecto, cuyo efecto colateral implica la existencia de un mayor número de recursos online y una mayor variedad de complementos que simplifican el desarrollo del front-end.

\subsection{Docker}

Docker \cite{tech:docker} es un software de código abierto enfocado al despliegue automatizado de aplicaciones que se basan en arquitecturas de microservicios. Su funcionamiento consiste en la encapsulación de los distintos servicios en contenedores los cuales funcionan de manera similar a máquinas virtuales, aislando cada componente en su propio sistema operativo. 

\subsection{CSS}

CSS (Cascading Style Sheets) \cite{lang:css} es un lenguaje de diseño gráfico conformado por hojas de estilo en cascadas cuyo objetivo es definir la apariencia visual de los documentos de HTML o XML. Mediante CSS se pretende mantener de manera independiente el contenido de uno o varios documentos de su presentación facilitando la separación de responsabilidades entre los distintos componentes de una página web.
El uso de CSS en el proyecto ha sido muy importante ya que se ha tratado de crear una plataforma intuitiva y estilizada de acuerdo con los estándares actuales que se pueden encontrar en cualquier plataforma. 

\subsection{Bash}

Bash \cite{wiki:bash} es un lenguaje de scripting utilizado en los sistemas operativos Unix para la ejecución de ordenes por medio de una terminal. Mediante Bash es posible lanzar multitud de comandos disponibles en sistema operativo que permiten controlar cualquier aspecto de este. El uso de Bash en proyecto se debe a la creación de script que permiten el arranque de los diferentes servicios que lo componen desde el interior de los contenedores Docker.

\section{Dependencias} \label{sec:dependencias}

\subsection{Python}

\subsubsection{PyGitHub}

PyGitHub \cite{dependencies:pygithub} es una biblioteca de Python que permite simplificar el proceso de establecer una conexión con la API REST de GitHub. Mediante su sistema de clases permite acceder y modificar los atributos de los diferentes elementos a los que GitHub permite su acceso sin tener que lidiar con peticiones HTTP. Otra de sus ventajas es el tratamiento de errores que realiza, que traduce los fallos de las peticiones HTTP en excepciones Python abstrayendo por completo la lógica del canal de comunicación.

\subsubsection{SQLAlchemy}

SQLAlchemy \cite{dependencies:sqlalchemy} es una biblioteca de Python que actúa como Object Relactional Mapper (ORM) permitiendo mapear las estructuras de una base de datos relacional a una estructura lógica de entidades y otorgando un alto nivel de abstracción entre la base de datos y la lógica de la aplicación. Esta abstracción facilita al desarrollador no centrarse en el detalle de las múltiples variaciones que poseen las diferentes implementaciones de SQL por parte distintos sistemas gestores de bases de datos (SGBD).

\subsubsection{Flask}

Flask \cite{dependencies:flask} es un microframework desarrollado en Python con el objetivo de simplificar el desarrollo de aplicaciones web siguiendo una arquitectura de Modelo-Vista-Controlador (MVC) \cite{tech:mvc}. Pese a que las funcionalidades base de Flask puedan parecer limitadas, esto es una decisión de diseño, que pretende mantener un paquete básico de funcionalidades y proporcionar a los desarrolladores escoger la extensión de sus funcionalidades a través de complementos.

El principal motivo de escoger Flask frente a su principal alternativa en Python, el framework Django, se recae en el uso para el cual se proponía el uso de estos paquetes. Pese a que Django es un complemento más completo para el desarrollo web, Flask suplía las necesidades de generar una API REST sencilla que atendiese las peticiones por medio de endpoints.

\subsubsection{Flask-CORS}

Flask-CORS \cite{dependencies:flask_cors} un complemento de Flask que permite habilitar el intercambio de recursos de origen cruzado (CORS) por medio del uso de cabeceras adicionales. Esta funcionalidad capacita a nuestra web comunicarse con el back-end pese a encontrarse desplegados en servidores de orígenes (dominios) distintos.

\subsubsection{PyTorch}

PyTorch \cite{dependencies:pytorch} es una de las bibliotecas de aprendizaje automático más populares disponible para tanto para Python y como C++ basada en Torch. Su principal ventaja se encuentra en la computación por medio del uso de tensores, lo cual habilita la ejecución de operaciones que requieren de un alto coste computacional por medio de la GPU. Los tensores son una abstracción de los vectores numéricos que soportan múltiples dimensiones y sus respectivas operaciones. Su popularidad se debe en parte al robusto ecosistema de herramientas construido a su alrededor ha potenciado su uso en aplicaciones de aprendizaje profundo entre las que se encuentran el procesamiento del lenguaje natural o la visión artificial (computer visión).

\subsubsection{Transformers}

Transformers \cite{tech:transformers} es una biblioteca desarrollada por Hugging Face que proporciona el acceso a arquitecturas de propósito general modelos para la resolución de tareas de procesamiento natural a través de modelos previamente entrenados. Esta biblioteca se diseño inicialmente como haciendo uso de PyTorch, aunque en la actualidad permite al usuario hacer uso de las capacidades de las bibliotecas de TensorFlow diseñadas por Google. Dependiendo del tipo de tareas de procesamiento y generación de lenguaje natural con las que estemos buscando trabajar, Transformers pone a disposición de los desarrolladores más de 32 modelos entrenados en más de 100 lenguajes distintos. Esta abrumante cantidad de modelos disponibles para su aplicación de manera directa, sin necesidad de configuraciones o ajustes convirtieron al paquete en la opción escogida al cual confiar el procesamiento de las tareas de PLN lanzadas a través de nuestra plataforma.

\subsubsection{Numpy}

Numpy \cite{dependencies:numpy} es una biblioteca de Python que proporciona soporte para la creación de vectores multidimensionales y sus operaciones. Su uso es prácticamente imprescindible en proyecto Python que requiera de cálculos matemáticos sobre grandes volúmenes de datos debido a su alto rendimiento y facilidad de uso.
Su eficiencia viene determinada por la manera en la que trabaja con los datos a través de su implementación propia de los arreglos “ndarray”. A expensas de perder la flexibilidad de los arreglos base de Python, esta implementación solo permite almacenar elementos de un único tipo por arreglo. Esta limitación es la que permite que los elementos sean almacenados de manera contigua en memoria, lo que agiliza en gran medida la ejecución de operaciones matemáticas de elevada complejidad.

\subsection{React}

\subsubsection{Wouter}

Wouter \cite{dependencies:wouter} es una biblioteca minimalista de enrutamiento que se plantea como una alternativa al popular paquete React Router implementando exclusivamente sus funciones básicas más utilizadas, y descartando aquellas funciones que provocan que React Router sea una dependencia mucho más pesada. Los elementos incluidos permiten renderizar componentes de manera condicional a la ruta accedida y manipular la ubicación del navegador a través su envoltorio para la API del historial del navegador.
La motivación para incluir esta biblioteca viene determinada por la necesidad de un complemento que facilitase el renderizado las diferentes secciones de la aplicación en función de la dirección a la que se dirigiese el usuario, así como una manera redireccionar el usuario a diferentes rutas en función de la información introducida por medio de los formularios dispuestos en la web.

\subsubsection{React Google Charts}

React Google Charts \cite{dependencies:react_google_charts} es una biblioteca para React que permite renderizar gráficos de manera sencilla actuando como un \emph{wrapper} de la \emph{Google Visualization API}. Su implementación tan solo requiere del uso del componente principal <<Chart>>, al cual se le suministran los datos y las opciones de personalización que se deseen. La selección de este complemento frente a otras alternativas más populares como \emph{ChartJS} viene dada por el hecho de que tras numerosas pruebas el complemento escogido requiere de una menor configuración para lograr una visualización de los datos adaptada al tipo de representación deseada.

\section{Herramientas} \label{sec:herramientas}

\subsection{Visual Studio Code}

Visual Studio Code \cite{tech:vscode} es un popular editor de texto de código abierto especializado en la edición de código fuente desarrollado por Microsoft. Entre sus principales características destacan la integración con el software de control de versiones Git, el resaltado de la sintaxis de una gran variedad de lenguajes de programación, un destacable control de sus funciones por medio de atajos de teclados y la inclusión de herramientas de depuración. Su mayor virtud frente a sus competidores es el amplio apoyo de la comunidad entorno a la creación de extensiones que permiten modificar, añadir y personalizar casi cualquier funcionalidad del editor.

\subsection{Codebeat}

Codebeat \cite{tech:codebeat} es una herramienta de análisis de código automático con soporte a múltiples lenguajes de programación. El objetivo es obtener un informe que evalúe la calidad del código por medio de la detección de vulnerabilidades, indicadores de code smell, duplicidad de código y complejidad. La plataforma detecta automáticamente los cambios en el repositorio y envía un aviso por correo electrónico de los resultados obtenidos. Es de gran utilidad ya que proporciona con un alto nivel de detalle la información relativa a la gravedad del problema y la ubicación exacta del fragmento de código que lo ha producido.

\subsection{Overleaf}

Overleaf \cite{tech:overleaf} es un editor de LaTeX online colaborativo que facilita la generación de documentación científica. Entre sus virtudes se encuentra la sugerencia de comandos LaTeX, la compilación de los documentos de manera automática y la posibilidad de mantener una vista previa actualizada de los cambios que se producen en el documento en tiempo real. Incluye indicadores que señalan los errores detectados en la compilación directamente en el editor y conserva un historial de los cambios realizados en el documento.

\capitulo{5}{Aspectos relevantes del desarrollo del proyecto}

\section{Arquitectura del proyecto} \label{sec:arquitectura}

\subsection{Arquitectura de Microservicios} \label{sec:microservicios}
La planificación de la arquitectura sobre la cual se iba a desarrollar el proyecto quedó definida en las primeras fases de exploración de la idea. Su elección estuvo condicionada al cumplimiento de una serie de características que se planteaban como clave en el desarrollo de una plataforma moderna. Entre las propiedades de las que se pretendía dotar a la aplicación destacan las siguientes:

\begin{itemize}
    \item [\textbullet] \textbf{Bajo acoplamiento}. Se decidió buscar la separación entre sí de aquellos componentes que no dependiesen del resto para su ejecución. Una de las principales ventajas de este tipo de planteamientos reside en minimizar el impacto que puede tener una falla que se de en alguna de las fases del procesamiento que se llevan a cabo en el back-end. Otra de sus ventajas implica en proporcionar la posibilidad de optimizar los recursos a las necesidades de cada fase, asignando una menor cantidad de recursos para aquellas tareas de bajo coste computacional, y distribuyendo esos recursos en aquellos procesos que requieran de una mayor potencia de cálculo y/o almacenamiento.
    \item [\textbullet] \textbf{Mantenible y adaptable a cambios}. La esencia del proyecto resalta el concepto de experimentar con la aplicación de modelos de PLN a los sistemas de seguimiento de incidencia con el objetivo de lograr simplificar y mejorar partes del proceso. Esta propuesta deja abierta la posibilidad de incorporar nuevas tareas y funcionalidades a la plataforma de cara a un futuro, así como la actualización de modelos a nuevas implementaciones.
    \item [\textbullet] \textbf{Alta disponibilidad}. La extracción de los datos y la aplicación de los modelos de PLN son procesos que demandan de una cantidad moderada de tiempo para la realización de sus funciones. Estos retrasos no deben repercutir en la experiencia del usuario, al cual se le debe seguir permitiendo interactuar con la plataforma mientras estos procesos se ejecutan en segundo plano.
\end{itemize}

Con el afán de cumplir con la implementación de una arquitectura que cumpliese dotase a la plataforma de las características anteriormente mencionadas nos decantamos por una arquitectura basada en microservicios. Este patrón arquitectónico permite encapsular las diferentes tareas de la aplicación en servicios independientes con un alto nivel de especificidad.

Una de las ventajas principales reside en el hecho de que la separación del servicio API REST encargado de gestionar las peticiones desde la aplicación, de los servicios de extracción o procesado, permite atender y encolar peticiones mientras se atienden peticiones previas evitando generar tiempos muertos. La segunda virtud de este tipo de arquitectura otorga a los mantenedores de la plataforma la opción de duplicar servicios en caso de que la demanda de este tipo de tareas supere a la capacidad de procesamiento de los servicios actuales.

\subsection{Arquitectura dirigida por eventos} \label{sec:dirigida_por_eventos}

La decisión arquitectónica escogida entorno a la separación de los diferentes servicios en componentes trae consigo la necesidad de implementar algún mecanismo mediante el cual los componentes sean capaces de detectar cuando deben llevar a cabo sus labores. El mecanismo requerido nos condujo directamente a la implantación de una arquitectura basada en eventos.
Este patrón arquitectónico suele aplicarse en entornos asíncronos, es decir, que cuando se genera una petición esta no siempre va a poder ser atendida y resuelta en ese mismo instante. Las peticiones entrantes son atendidas por un agente, este se encarga de emitir un evento y redireccionarlo a la cola de tareas del correspondiente servicio. Por su parte, los servicios tienen la responsabilidad de atender a dichos eventos y actuar según proceda manteniendo un orden de acuerdo con las prioridades establecidas. Seguidamente se presentan los detalles de la implantación de este patrón en nuestro proyecto:

\vspace{-0.5cm}
\begin{itemize}
\item [\textbullet] \textbf{Agente generador de eventos}. El agente generador de eventos de nuestra plataforma es el servicio de API REST, el cual se encarga de atender las peticiones lanzadas por los usuarios a través de la web. Las peticiones son procesadas, retornando una respuesta informando de si esta ha sido aceptada o rechazada debido a algún tipo de error en los parámetros requeridos.  Ante una petición correcta el sistema genera una nueva entrada en la base de datos detallando el tipo de tarea que ha sido solicitada, y los atributos asociados a dicha tarea.
\item [\textbullet] \textbf{Canal de comunicación de eventos}. La implantación del canal a través del cual los servicios son conscientes de las peticiones entrantes es una tabla de la base de datos donde cada fila representa una nueva tarea asociada a un identificador único. En los atributos de cada entrada es donde se indica hacia qué servicio va dirigida la petición, una bandera que indica el estado en el que se encuentra (esperando, en proceso, resuelta, fallida o desactualizada).
\item [\textbullet] \textbf{Consumidores de eventos}. Los servicios consumidores de eventos son tanto el servicio de procesamiento como el de extracción. Estos servicios acceden de manera periódica a la base de datos en busca de nuevas peticiones que requieran de su atención. En el momento se detecta un nuevo evento estos recuperan los argumentos de la petición y actualizan su estado. Finalmente, una vez resuelta la petición actualizan el estado de la tarea de acorde a su resolución.
\end{itemize}

El alto grado sobre el cual se ha implementado la aplicación no se encuentra libre de inconvenientes. Entre ellos debemos destacar la necesidad del agente externo que lanza las solicitudes contra la API REST de encargarse también de recoger los datos consultando a dicho servicio actualizaciones sobre el estado en el que se encuentra su petición. Este efecto se ve mitigado debido a que será el servidor que maneja la aplicación web el encargado de realizar llamadas de manera periódica hasta que vea resuelta su petición.

\capitulo{6}{Trabajos relacionados}

La popularidad de los algoritmos de PLN ha provocado una enorme avalancha de proyectos y trabajos con la intención de mejorarlos, ampliarlos, estudios y comparativas de eficiencia entre diferentes configuraciones de parámetros y el uso de diferentes analizadores léxicos, y la búsqueda de nuevas áreas de aplicación. Entre los proyectos previos que podemos encontrar en la universidad en relación con esta temática se destacan los proyectos dispuestos a continuación.

\section{Label prediction on issue tracking systems using text mining} \label{sec:labelpredictionarticle}
El artículo <<\emph{Label prediction on issue tracking systems using text mining}>> \cite{ related_works:label_prediction} es el punto de partida del proyecto \textbf{GitHub Text Mining}. Este ensayo tuvo como objetivo el estudio de la clasificación de incidencias por medio del uso de avanzadas técnicas de minería de datos y clasificación como son las \textbf{Máquinas de Vectores de Soporte (SVM)} y el clasificador \textbf{Naive Bayes Multinomial (NVM)}. Sus investigaciones se centraron en el entrenamiento de los clasificadores anteriormente mencionados y el estudio de la precisión de las predicciones obtenidas sobre las incidencias un conjunto discreto de repositorios de GitHub.

A raíz de los resultados logrados surgió la idea de diseñar una plataforma que permitiese automatizar el proceso de obtención de los datos y experimentar con la aplicación modelos de clasificación preentrenados. Estos modelos preentrenados permitirían eludir el complejo proceso de entrenar un modelo desde cero, así como conseguir comprender una variedad temática más amplia a costa de cierta pérdida de precisión en los resultados. A lo largo de la evolución del proyecto se decidió ampliar el tipo de tareas de PLN soportadas por la plataforma.

\section{JIZT}
\emph{JIZT} \cite{related_works:jizt} es una plataforma de generación de resúmenes abstractivos que hace uso de técnicas de procesamiento del lenguaje natural para lograr su cometido. El proyecto logró ofrecer unos buenos resultados mediante la aplicación de modelos de PLN y la incorporación de novedosas técnicas en el preprocesado de los datos. Estas innovaciones han sido incorporadas en nuestro proyecto con el objetivo de lograr mejorar los resultados en aquellas situaciones en las que los datos sobre los cuales se deseaba aplicar los modelos superaban las limitaciones de tamaño de estos.

\section{TFG Plataforma de text mining sobre repositorios de código abierto GitLab}
\emph{Este trabajo} \cite{related_works:tfg_text_mining_gitlab} se incluye con el objetivo de señalar otro enfoque distinto mediante el cual afrontar un proyecto que tiene como base el mismo punto de partida en el \hyperref[sec:labelpredictionarticle]{artículo anteriormente mencionado}. La plataforma presentada tendría como objetivo extraer de manera automatizada de datos de repositorios de \emph{GitLab} y permitir la clasificación de las incidencias de acuerdo múltiples estrategias implementadas en específico para la plataforma.
\capitulo{7}{Conclusiones y Líneas de trabajo futuras}

\section{Conclusiones}

El proyecto \textbf{GitHub Text Mining} es el resultado del esfuerzo invertido durante un plazo de aproximadamente \textbf{6 meses} con el ambicioso objetivo de lograr desarrollar todos los aspectos que hacen a una plataforma funcional desde cero. En este trabajo se llevado a cabo un desarrollo \textbf{full stack}, o lo que es lo mismo, se ha desarrollado tanto el aspecto interno de una aplicación, trabajando con \textbf{tecnologías de servidor}, como el desarrollo de aquellos aspectos relacionados con la \textbf{presentación de la información} y el \textbf{establecimiento de conexiones con la fuente de los datos}.

Las principal conclusión que se extrae del desarrollo de este proyecto es \textbf{la importancia de tomar una decisión y ser capaz de defenderla}. En un trabajo de tal magnitud es inevitable enfrentarse a momentos en los que afrontar decisiones que marcarán la dirección del proyecto a futuro. Estas decisiones son tomadas en base a los conocimientos que se poseen en dicho momento, y por lo tanto, es común que con el paso del tiempo y la adquisición de experiencia se formulen nuevas propuestas con la finalidad de perfeccionar las implementaciones en apartados previos. Es necesario establecerse límites en cuanto a \textbf{asumir que un proyecto nunca va a estar completo}, y consecuentemente ceñirse en sacar a delante los objetivos propuestos antes de tratar de actualizar aspectos que actualmente cumplen con su función.

La segunda conclusión extraída de este proyecto, directamente relacionada con la mencionada en el anterior párrafo, es \textbf{la importancia de invertir tiempo en estudiar la dirección que va a tomar un proyecto y las alternativas disponibles} antes de aferrarse por completo a una idea. En un proyecto de gran magnitud, invertir recursos en la investigación y desarrollo de una buena arquitectura que dote a un proyecto de robustez resulta un aspecto clave de cara a la incorporación de cambios y novedades en un futuro.

Para finalizar, se destaca la relevancia de afrontar este tipo de Trabajos de Fin de Grado como una oportunidad para \textbf{aplicar los conocimientos adquiridos} a lo largo del Grado, y aprovechar para indagar sobre aquellas áreas de interés en las que tal vez no se haya profundizado tanto como a uno le gustaría. \textbf{Aprovechar la libertad creativa del proyecto} es una forma de descubrir que facetas de la informática encajan con nuestra perspectiva y establecer nuestro próximo objetivo personal.

\section{Líneas de trabajo futuras}

En esta sección se incluyen ideas acerca de la incorporación de nuevas funcionalidades que quedaron descartadas debido al alcance establecido para el proyecto, así como posibles áreas de mejora sobre las que seguir trabajando.

\subsection{Mejoras e incorporación de nuevas tareas}

Desde el inicio el proyecto GitHub Text Mining se planteó con la idea de establecer una arquitectura base que permitiese la ampliación de los modelos soportados a otro tipo de tareas. Actualmente los experimentos que se incluyen permiten obtener una clasificación de las incidencias entorno a etiquetas, generar resúmenes a partir su contenido y realizar un análisis de sentimientos sobre las incidencias, incluyendo filtros de usuarios y comentarios. En adición a la introducción de mejoras en las respectivas implementaciones de estas tareas, se plantea la introducción de nuevas funcionalidades que permitan detectar incidencias duplicadas en un repositorio haciendo uso de modelos de \textbf{Sentence Similarity}, así como estudiar la introducción de tareas de \textbf{Traducción}.

\subsection{Implementar acceso mediante el uso de tokenes de acceso personal por usuario}

Una decisión de diseño tomada inicialmente condiciona a la utilización de un único token de acceso personal para el establecimiento de la conexión con la API de GitHub. Este factor puede resultar limitante en caso de que se pretenda desplegar la aplicación a causa de las \textbf{limitaciones de la API de GitHub} \cite{lf:github_api_limits}. Sería preferible diseñar alguna estrategia alternativa en la que cada usuario deba proporcionar su propio token para poder utilizar la plataforma. Se deberá tener en cuenta que estos tokenes son información personal, y por lo tanto deberán transferirse, tratarse y almacenarse de forma segura.

\subsection{Refractorización de la API REST}

La \emph{API REST} ha ido creciendo conforme se requerían nuevas formas de acceder a los datos desde la aplicación web. Esto ha propiciado la inclusión de una gran cantidad de métodos que deberían ser refractorizados con el objetivo de reducir su complejidad y deshacerse de aquellos \emph{endpoints} que resulten prescindibles. Otro punto de mejora a considerar implicaría la inclusión de un sistema que estableciese limitaciones en el número de peticiones que se permiten recibir desde una misma dirección durante un cierto periodo de tiempo.


%Where the bibliography will be printed
  \printbibliography

\end{document}
